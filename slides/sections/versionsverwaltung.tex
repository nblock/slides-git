\section{Versionsverwaltung}

\begin{frame}
  \tableofcontents[currentsection]
\end{frame}

\begin{frame}{Was ist Versionsverwaltung}
Aus der Wikipedia:

Eine Versionsverwaltung ist ein System,
das zur \textbf{Erfassung von Änderungen an
Dokumenten oder Dateien verwendet wird}.
Alle Versionen werden in einem \textbf{Archiv} mit
Zeitstempel und Benutzerkennung gesichert und können
später \textbf{wiederhergestellt} werden.
Versionsverwaltungssysteme werden typischerweise in der
Softwareentwicklung eingesetzt um Quelltexte zu verwalten.
Versionsverwaltung kommt auch bei Büroanwendungen oder
Content-Management-Systemen zum Einsatz.
\end{frame}

\begin{frame}{Gründe für die Verwendung von Versionsverwaltung}
  \begin{itemize}
    \item Wer hat wann und warum Änderungen vorgenommen
    \item Snapshots von unterschiedlichen Zuständen der Software
    \item Unterschiede zwischen Versionen ermitteln
    \item Einfaches Teilen von Ergebnissen
    \item Änderungen anderer in die eigene Arbeitskopie integrieren (merging)
    \item Bestimmte Versionen festschreiben (tags)
    \item Neue Ideen ausprobieren
    \item Good/Best practice
  \end{itemize}
\end{frame}

\begin{frame}{Gründe gegen die Verwendung von Versionsverwaltung}
  \begin{itemize}
    \item Backups?!
  \end{itemize}
\end{frame}

\begin{frame}[allowframebreaks]{Versionsverwaltungs-Arten}
  \begin{itemize}
    \item Lokale Versionsverwaltung
    \begin{itemize}
      \item Single-User
      \item System kennt keine anderen Arbeitskopien oder Benutzer
      \item Mittlerweile kaum noch Bedeutung
      \item Beispiel: rcs
    \end{itemize}
    \framebreak

    \item Zentrale Versionsverwaltung
    \begin{itemize}
      \item Ein ,,heiliges'' Repository für alle Benutzer
      \item Repository meist auf einem Server
      \item Jedes Einchecken von Änderungen benötigt Zugriff auf das Repository, also Netzwerkverbindung
      \item Forks sind extrem aufwändig
      \item Beispiele: CVS, Subversion
    \end{itemize}
    \framebreak

    \item Dezentrale Versionsverwaltung
    \begin{itemize}
      \item Arbeitskopie beinhaltet automatisch Repository
      \item Repositories können im Netz veröffentlicht werden
      \item Veröffentlichte Repositories können geklont werden
      \item Jede Arbeitskopie ist per Definition ein Fork
      \item Beispiele: git, Bazaar, Mercurial, Monotone
    \end{itemize}
  \end{itemize}
\end{frame}

% vim: tabstop=2 expandtab shiftwidth=2 softtabstop=2 autoindent
